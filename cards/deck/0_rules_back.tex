\begin{tikzpicture}
  \cardborder
  \cardtype{rulesbg}{Histoire de l'informatique}
  \cardcontent{}{À chacun son tour, un joueur place une carte de son choix à ce qu'il pense être sa bonne position dans la frise.\\
  Une fois posée, la carte est retournée : si elle est en bonne position, la carte est laissée face visible pour compléter la frise, sinon, la carte est défaussée et le joueur en tire une nouvelle.\\
  Le premier joueur a avoir placé correctement toutes ses cartes est déclaré vainqueur. Si la pioche est épuisée alors qu'un joueur doit tirer à nouveau, il est déclaré perdant.\\
  \emph{Les descriptions sont issues de \texttt{fr.wikipedia.org}.}
  \begin{flushright}
  Romuald \textsc{Thion}\\
  Département informatique\\
  Faculté des sciences et technologies\\
  Université Claude Bernard Lyon 1
  \end{flushright}
  }
\end{tikzpicture}
